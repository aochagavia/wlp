\documentclass[a4paper]{article}

\usepackage{xltxtra}
\usepackage{url}
\usepackage{polyglossia}
\usepackage{amsmath}
\setdefaultlanguage{english}
\usepackage{xspace}
\usepackage{fullpage}

% For displaying pseudocode/GCL
\usepackage{algorithm}
\usepackage{algpseudocode}
\newcommand{\Assert}{\State\textbf{assert}\xspace}
\newcommand{\Assume}{\State\textbf{assume}\xspace}
\algblockdefx{Var}{EndVar}[1]{\textbf{var} #1 \textbf{in}}{\textbf{end var}}

% Command to typeset ++
\newcommand\mdoubleplus{\ensuremath{\mathbin{+\mkern-5mu+}}}

\author{
  Tim Baanen\\4134524
  \and
  Adolfo Ochagavía\\4045483
}
\title{WLP Verification Tool}

\begin{document}

\maketitle

\section{Introduction}

Within the field of software verification, there are different approaches to
increase confidence in the correctness of a program. This project focuses on
the testing approach, which gives up completeness for the sake of simplicity.

Saying that this project takes a testing approach is not precise,
since many methodologies coexist in the field of automated software testing. A
relatively recent overview of the current state of affairs is presented by Ammann
\cite{ammann2008introduction}. In this project, we generate and run test cases
based on models of programs. Said models not only reflect the behavior of a
given program, but also provide assumptions and assertions that should be respected.

\section{The Verificaction Tool}

The verification tool implemented in this project models programs using a
variant of Dijkstra's Guarded Command Language (GCL) \cite{Dijkstra:gcl}.
The language supports assertions, which are treated as postconditions,
and assumptions, treated as preconditions.

Based on the GCL model of a program, our tool performs the following steps:

\begin{enumerate}
\item Calculate the possible execution paths up to a given length (performed by
the \texttt{path} function in \texttt{Lib.hs}).
\item For each path, calculate the Weakest Liberal Precondition (WLP) % TODO: cite a source here?
taking assumptions and assertions into account (performed by the \texttt{wlp'}
and \texttt{wlpPath} functions in \texttt{Lib.hs}).
\item For each WLP, calculate a QuickCheck \cite{claessen2011quickcheck}
generator of test cases (performed by the \texttt{testPredicate} function in
\texttt{Lib.hs}).
\item For each \texttt{Property}, feed it to QuickCheck so it can be tested (see
\texttt{Spec.hs} for examples).
\end{enumerate}

The subsections below explain in detail each of the previous steps, except the last
one.

\subsection{Calculating execution paths}

Since our modeling language supports control flow constructs, a program may have
multiple execution paths depending on the initial values of the variables and
parameters. Furthermore, programs may loop indefinitely and therefore can be
traversed by execution paths of infinite length. Given the undecidability of the
halting problem, we introduce an artificial length limit that a path is not
allowed to exceed.

When calculating execution paths, we insert \texttt{assume} and \texttt{assert}
statements in order to drive the WLP calculation later on. Since they are not
really part of the execution path, they are ignored when determining its length.

Some constructs of the language are unrelated to control flow and therefore have
a trivial implementation of the $path$ function. This is the case for
\texttt{Skip}, \texttt{Assert}, \texttt{Assume} and \texttt{Assign}. Other
constructs, however, affect control flow or at least contain other constructs
that do. Below is a description of how they are handled:

\subsubsection*{Variable declaration}

According to the grammar of GCL, variable declaration is always followed by a
statement in its scope. Therefore, we need to calculate the execution paths for
the given statement and append the variable declaration to each of them.

\subsubsection*{Sequence}

\texttt{Sequence} combines two statements in such a way that one is executed
after the other. This means we need to combine the paths of either statement
to get the path through the sequenced statements. Formally, the possible
execution paths for \texttt{Sequence} are defined as:
\begin{align*}
	\text{Paths}(\text{first} ; \text{second}) &= \{ p_1 \mdoubleplus p_2 \mid p_1 \in \text{Paths}(\text{first}), p_2 \in \text{Paths}(\text{second})\}.
\end{align*}

\subsubsection*{If-then-else}

The if-then-else statement contains two statements, one for the \texttt{then} branch
and one for the \texttt{else} branch. Since only one of them will be executed
when the program is run, the paths for if-then-else are defined as:
\begin{align*}
	\text{Paths}(\texttt{if } \text{cond } \text{then } \text{else}) &= \text{Paths}(\text{then}) \cup \text{Paths}(\text{else}).
\end{align*}

Note, however, that we need to prepend an assumption of the condition for each
path in the \texttt{then} branch, and a negation thereof for each path in the
\texttt{else} branch. This is necessary to drive the WLP calculation later on.

\subsubsection*{While}

If the while loop is annotated with an invariant, we can use it directly in the
WLP calculation. However, if the loop is not annotated, we must supply our own
precondition for the given postcondition. To do this using a list of program
paths, we consider two situations:

\begin{itemize}
\item The condition is initially false
\item The condition is initially true, so the body and a new path through
the while loop are executed
\end{itemize}

By doing this recursively, we can obtain the paths in the body and unroll the
loop as much as possible:
\begin{align*}
	\text{Paths}(\texttt{while } \text{cond } \text{body}) &= \{[]\} \cup (\text{Paths}(\text{body}) \mdoubleplus \text{Paths}(\texttt{while } \text{cond } \text{body}) ).
\end{align*}

Of course, in our code we add a path length limit in order to prevent generation
of infinite paths. Additionally, in the case when the condition is initially false,
the body of the while is not executed so the path is empty. As in the
if-then-else paths, we use assumptions to control the state of the condition.

Note that we do not do a finite unfolding where we assert that the loop is
traversed a set number of times. Instead, we assume the number of loop
iterations for each path. If we used assertions instead of assumptions, the
checker would reject all loops. Given two paths with different iteration counts,
no (consistent) precondition would satisfy an assertion that both paths will
be executed, but there are preconditions that either path may be taken.

\subsubsection*{Program calls}

We handle program calls by including the code of the called program in a path.
The case of recursive calls is handled nicely by a combination of Haskell's
lazy evaluation semantics and the upper limit on the paths, which ensures no
infinitely long path will need to be generated.

However, we cannot directly inline a program that contains pre- and
postconditions. The calling function has the responsibility to satisfy these
conditions, and cannot simply pass the responsibility to its own caller.
Instead, we drop all assertions and assumptions in the called program. Note
that we do keep the assumptions generated by generation of program paths.

\subsection{Calculating WLP}

The implementation of WLP for \texttt{Skip}, \texttt{Sequence}, \texttt{Assume}
and \texttt{Assert} is either trivial or recursively defined. However, the
implementations for \texttt{Assign} and \texttt{Var} are less straightforward.

\subsubsection*{Var}

The WLP of a \texttt{Var} statement is equivalent to the WLP of its inner
statement up to the introduction of new variables. This seems trivial at
first, but we need to consider the fact that the declared variables may be
shadowing already existing names. In order to avoid that, we rename all declared
variables that shadow existing names in such a way that they get a unique one.
This is done by the \texttt{refresh} function (defined in
\texttt{Rewriting.hs}). Then we apply the $\forall$ quantifier to any remaining
variables.

% Idea: write an appendix telling more about how refresh works?

\subsubsection*{Assign}

The WLP of assignment involves replacing names by expressions in the postcondition.
When the target of the assignment is a name, we replace the given name by the value
of the assignment. When the target is an array, we replace the indexed
assignment by an appropriate \texttt{Repby} expression. Refer to the
\texttt{replace} function for the concrete implementation details of the
replacing algorithm (defined in \texttt{Rewriting.hs}).

We also considered the fact that simultaneous assignment to array indices does
not work like simultaneous assignment to variables. If the statement 
$a[i], a[j] \gets a[j], a[i]$ has similar semantics to $x, y \gets y, x$, the
assignment needs the intermediate result of
$a \gets \texttt{Repby}\ a, i, a[j]$ to get to the correct result
$a \gets \texttt{Repby}\ (\texttt{Repby}\ a, j, a[i]), i, a[j]$.

\subsubsection*{While with invariant}

The WLP of a while loop with an invariant is the invariant itself, provided the
invariant is a valid one in the context of the postcondition. However, checking
this constraint requires that we actually test the body of the loop, which
would make the tester depend on itself recursively. Therefore, we assume the
user supplies the correct invariant.

\subsection{Calculating QuickCheck properties}

The output of the WLP step is a GCL expression that should evaluate to a boolean
after instantiating the free variables. Constructing a QuickCheck \texttt{Property}
based on the WLP is achieved through the following steps:

\begin{enumerate}
\item Normalize the WLP
\item Calculate the ranges
\item Generate sets of variable instantiations
\item Ensure that the WLP holds for all sets of variable instantiations
\end{enumerate}

A detailed explanation of the workings of each step is provided below.

\subsubsection*{Normalize WLP}

We rewrite the predicate in the prenex normal form. After normalization, the
predicate will start with a string of quantifiers, followed by a quantifier-free
formula. Since we find a counterexample to the predicate by checking various
instantiations of the quantifiers, normalizing will show exactly which
quantifiers need to be instantiated.

By discarding all universal quantifiers in front of the WLP, we do not have to
handle these separately, but as free variables. A formula without quantifiers
and instantiations for all free variables is very easy to decide, while many
formulas with quantifiers are undecidable.

Additionally, we do some rewriting on the predicate to simplify it. These steps
include removing double negations and reducing expressions with constants,
such as $\phi \vee \bot$, to simpler expressions such as $\phi$.

\subsubsection*{Calculate Ranges}

We need QuickCheck to generate only relevant test cases. That is, test cases that
are likely to make the program fail. When the precondition does not hold
for the example we want to consider, there is no way to make the program fail.
Similarly, when the postcondition always holds, there is no need to search for
preconditions that make the postcondition fail. Therefore, we consider a test
case irrelevant when it satisfies either:

\begin{itemize}
\item Invalid precondition
\item Trivial postcondition
\end{itemize}

Therefore, we introduce the concept of ranges. By restricting a variable to a
well-chosen range, we need to consider less test cases but still receive an
equivalent predicate. Initially, all free variables get an infinite range, which
is then partially narrowed. Because complete narrowing would be equivalent to
SAT solving, we only consider relatively simple cases.

As an example, let us consider the ranges that would be generated for the following WLP
expression:
\begin{align*}
	((0 \leq k \leq 10) \wedge (x = y) \wedge p) \Rightarrow (q \wedge \exists n (1 \leq n \leq 5)).
\end{align*}

It is clear that we want to generate a $k$ in the range $[0, 10]$, the value of
$x$ to be equal to $y$ and a $p$ that is \texttt{True}. In any other case, the
left side of the implication cannot be true and the whole expression will
trivially evaluate to \texttt{True}. Moreover, we want $q$ to be \texttt{False}
to find a counterexample on the right side of the implication. Note however that
the range for $n$ is $[1, 5]$, since it suffices to (fully) check this range to
find a contradiction. The function \texttt{nonTrivRange} calculates these
ranges. Since many preconditions involve variables depending on other variables,
we also implemented a function \texttt{nonTrivAlias} that checks when two
variables should have exactly the same instantiations.

From the ranges we have constructed for the variables, we make QuickCheck
generators that generate values in the corresponding ranges. If the range turns
out to be empty for some variable, QuickCheck will raise an error. Therefore
we need to check that the ranges are non-empty before turning them into generators.

\subsubsection*{Ensure that the WLP holds for all sets of variable instantiations}

In this step we construct a \texttt{Property} that takes the input generated by
the previous generator, feeds it to the WLP and ensures that it holds.

% This paragraph is correct (there used to be a fixme because I didn't know for sure)
In case any of the ranges is empty, we know that the precondition cannot be
satisfied or the postcondition is always satisfied. This means that the
\texttt{Property} always holds.

However, in most cases, the ranges are not empty and we try to find a counterexample
based on the generated instantiations. That is, for each generated set of variable
instantiations, we test to see if the calculated WLP holds.

The test is performed by the \texttt{runCase} function, which returns \texttt{True}
when the WLP is found to be valid and \texttt{False} otherwise. The function
operates in two steps:

\begin{enumerate}
\item Ensure that all expressions in an assignment target are evaluated (performed
by the \texttt{literalize} function). This ensures array indexing is performed
consistently, so $a[i]$ and $a[j]$ will always be the same value if $i = j$.
\item Evaluate the WLP with the given variable instantiations (performed by the
\texttt{evaluateClosed} function in \texttt{Eval.hs}).
\end{enumerate}

When evaluating the WLP expression there are three possibilities:

\begin{enumerate}
\item The expression starts with a universal quantifier or has free variables:
we try to instantiate the variables to find a counterexample. If we find one, we
are sure the predicate does not always hold. If we do not find one, we can only
conclude that there probably is no counterexample that can be found easily.
\item The expression starts with an existential quantifier: we try to instantiate
the variables to find an example. Dually to the universal quantifier, we can only
be sure of finding an example, and we interpret the failure in a weak sense.
\item There are no quantifiers: this means that all variables should have been
instantiated. In this case, we directly evaluate the WLP for the given
instantiations, and be sure about the outcome.
\end{enumerate}

Unfortunately, QuickCheck is designed around universal quantification, which means
that it provides no straightforward way to deal with existential quantifiers.
For this reason, we had to extend QuickCheck with additional types and functions
in order to meet our needs. These extensions are defined in \texttt{Checker.hs}.

\section{Results}

We applied the checker to various programs to determine how effective our design
is on practical problems. We define a path to be tested \emph{adequately} when
it is tested with a relevant case. In the ideal case, every program path should
be tested adequately. With these test programs, we want to show that the
concessions we made do not result in practical problems in reducing the adequacy.

\subsection{Loops and assignment: example E}

The first program we tested our verifier on was Program \ref{exampleProgram}.
While it has no practical use, it contains the basic loop and assignment
structure that we designed our verification tool for.

\begin{algorithm}
\caption{Example program E} \label{exampleProgram}
\begin{algorithmic}
\Assume $-1 \leq x$
\While {$0 < x$}
	\State $x \gets x - 1$
\EndWhile
\State $y \gets x$
\Assert $y = 0$
\end{algorithmic}
\end{algorithm}

Note that the program does not have a valid precondition: if $x$ is equal to
$-1$, the while-loop is bypassed and so $y$ becomes $-1$. Our tool correctly
reports a failure with the output:\par
% some fancy formatting  for output with a lot of special characters.
\verb|Failed: Counterexample ((-1 <= x) /\ (x <= 0)) => (x == 0) (fromList [(x,-1)])|,\\
which indicates the WLP predicate $-1 \leq x \leq 0 \implies x = 0$ has the
counterexample $x = -1$. Changing the precondition to $0 \leq x$ causes the
verifier to correctly accept the program.

\subsection{Array indexing: minind}

To check that array indexing works as expected, we tested our verifier against Program \ref{minind}.

\begin{algorithm}
\caption{minind} \label{minind}
\begin{algorithmic}
\Assume $N > i \wedge i' = i$
\Var {min}
	\State $\textit{min}, r \gets a[i], i$
	\While {$i < N$}
		\If {$a[i] < min$}
			\State $\textit{min}, r \gets a[i], i$
		\EndIf
		\State $i \gets i+1$
	\EndWhile
\EndVar
\Assert $\forall j\ [i' \leq j < N \implies a[j] \leq a[r]]$
\end{algorithmic}
\end{algorithm}

We inserted an incorrect postcondition in this program. In fact, the correct condition is $\forall j\ [i' \leq j < N \implies a[j] \geq a[r]]$. However, this distinction is only apparent when $i < N - 1$, so there are still program paths that result in a WLP that always holds. The verifier can correctly find a counterexample \texttt{(fromList [(N,7), (a,[RangeInt [(MinInfinite,MaxInfinite)]]), (i,5), (i',5), (j,5), (a[i],-27), (a[j],-17), (a[i + 1],-18)])}. The correct postcondition gets accepted by the verifier.

\subsection{Array assignment: swap}

Assignment to arrays requires us to correctly implement assignment to arrays at given indices. We made two variants of a program that swaps the values in an array at two given indices. The first, Program \ref{swapTemp}, uses an intermediate value and the second, Program \ref{swapSimultaneous}, uses a simultaneous assignment.

\begin{algorithm}
\caption{Swap using a temporary variable} \label{swapTemp}
\begin{algorithmic}
\Assume $a[i] = x \wedge a[j] = y$
\Var {\textit{tmp}}
\State $\textit{tmp} \gets a[j]$
\State $a[j] \gets a[i]$
\State $a[i] \gets \textit{tmp}$
\EndVar
\Assert $a[i] = y \wedge a[j] = x$
\end{algorithmic}
\end{algorithm}

\begin{algorithm}
\caption{Swap using simultaneous assignment}
\label{swapSimultaneous}
\begin{algorithmic}
\Assume $a[i] = x \wedge a[j] = y$
\State $a[i], a[j] \gets a[j], a[i]$
\Assert $a[i] = y \wedge a[j] = x$
\end{algorithmic}
\end{algorithm}

Since these programs contain only one path, we can easily calculate the WLP: $(a[i] = x \wedge a[j] = y) \implies (a[i] = x \wedge \texttt{Repby}\ (\texttt{Repby}\ a, j, a[i]), i, a[j])[i] = y)$. Our program automatically determines that $a[i]$ should have the same value as $x$ and $a[j]$ the same value as $y$ to satisfy the precondition.

\subsection{Program call: sort}

We also applied our verifier to a program that called other programs. To be more specific, the selection sorting program of Program \ref{sort} calls corrected versions of Programs \ref{minind} and \ref{swapTemp}.

\begin{algorithm}
\caption{Selection sort}
\label{sort}
\begin{algorithmic}
\Assume $N \geq 0$
\Var {$i$}
	\State $i \gets 0$
	\While{$i < N - 1$}
		\State $m \gets $\Call{minind}{$a, i + 1, N$}
		\If{$a[m] < a[i]$}
			\State $a \gets $\Call{swap}{$a, i, m$}
		\EndIf
	\EndWhile
\EndVar
\State $a' \gets a$
\Assert $\forall i [0 \leq i \leq N-1 \implies a'[i] \leq a'[i+1]]$
\end{algorithmic}
\end{algorithm}

This was a good test for our verifier, since it contains a lot of variable bindings. During development, we ran into many issues with bound variables that had the same name as different bound variables. Since the paths through this program get resolved to nested declarations of variables and we deliberately included a postcondition with the same variable, we can check that refreshing variables does work.

When we inspect the predicates generated by program paths, we can see a large number of them are trivially true. This is caused by the generation of program paths generating inconsistent paths. There are many cases where the outer while loop asserts $N$ to be greater than a given value to continue looping, but the inner loop of \texttt{minind} asserts $N$ to be smaller than the same value to break out of the loop. This is one of the major weaknesses of generating program paths for testing.

However, our verifier attempts to mitigate these issues by automatically detecting these trivial cases as described before. This leaves only the non-trivial cases to test. We checked that all non-trivial cases up to program length 50 are tested adequately by manually inspecting instantiations for all variables, and making sure these instantiations are adequate.

\subsection{Loop with invariant: minindInv}

\begin{algorithm}
\caption{minindInv} \label{minindInv}
\begin{algorithmic}
\Assume $N > i \wedge i' = i$
\Var {min}
	\State $\textit{min}, r \gets a[i], i$
	\While {$i < N$ \textbf{invariant} $\forall j\ [i' \leq j < i \implies a[j] \geq a[r]]$}
		\If {$a[i] < min$}
			\State $\textit{min}, r \gets a[i], i$
		\EndIf
		\State $i \gets i+1$
	\EndWhile
\EndVar
\Assert $\forall j\ [i' \leq j < N \implies a[j] \geq a[r]]$
\end{algorithmic}
\end{algorithm}

We modified Program \ref{minind} to use a while loop with invariant. Since the WLP of a while loop with a correct invariant is simply the invariant itself, and the preoondition $i' = i$ together with the range of quantification $i' \leq j < i$ results in a trivial test case. This shows the limitations of our design choices. However, incorporating Program \ref{minindInv} as a subroutine of the sorting program \ref{sort} reduced the program paths to non-trivial ones. Therefore, using an invariant is still a useful strategy to reduce test load with more paths.

\subsection{Quantification: findNonzero}

The final program that we tested the verifier on is Program \ref{findNonzero}.

\begin{algorithm}
\caption{findNonzero} \label{findNonzero}
\begin{algorithmic}
\Assume $\exists j [a[j] \neq 0]$
\While {$a[i] = 0$}
	\Var {$j$}
		\If {$a[j] \neq 0$}
			\State $i \gets j$
		\EndIf
	\EndVar
\EndWhile
\Assert $a[i] \neq 0$
\end{algorithmic}
\end{algorithm}

We used this program as a check that the verifier could correctly reason about quantifiers and the use of free variables without initialization. Since an unitialized free variable can take arbitrary values, we need to insert a $\forall$ quantifier. This requires that our program can also reason about existential quantifiers since the path through the if statement becomes an assertion. In fact, after the last normalization step, the predicate to be checked reduces to a trivial truth. Therefore, we conclude that our normalization function is effective even for complicated predicates.

\section{Related Work}

In this project we take a testing approach in which test cases are generated
randomly. An interesting approach similar to ours has been tried by Near and Jackson
\cite{rubicon}, who perform bounded verification of Ruby on Rails applications
using a tool called Rubicon. However, instead of using QuickCheck as a backend,
they use Alloy to do the checking \cite{alloy}. Therefore it should more
properly be considered a case of bounded verification instead of testing.

Another popular alternative for software verification are SMT solvers.
A good example of practical applications of
SMT solvers is presented by Lahiri and Qadeer \cite{Lahiri}, who verify correct
memory management of C programs based on a weakest precondition calculation.

An interesting technique associated to the world of testing is fuzzing \cite{takanen2008fuzzing}.
One of the main benefits of fuzzers is that they are highly automated and
produce remarkable results. For instance, American Fuzzy Lop (AFL) has a "trophy case"
with more than 100 bugs found in popular open source software such as PHP, nginx, QEMU
and VLC \cite{afl}. However, a limitation of fuzzing is that it is designed to find
inputs that make programs crash, instead of verifying that certain properties hold.

Finally, D'silva et al present a survey of automated techniques for formal software
verification, which may be of interest for the reader \cite{d2008survey}.

\section{Conclusion}

This project resulted in a verification tool that is sound but incomplete.
This is no surprise, since verifying soundly and completely is undecidable.
In practice, however, the results show that the tool is capable of finding
counterexamples in basic programs.

When developing the tool, it turned out that the range inference algorithm
was more complex than we had expected. We ended up choosing deliberately to
keep the implementation simple, which resulted in more work for QuickCheck.
Fortunately, our approach worked out very well, as can be seen in the results.

Another challenge we faced was handling variable renaming correctly. Initially,
we thought our implementation was correct. However, bugs started to pop out when
working on other features, which resulted in puzzling debugging sessions.

Finally, helping QuickCheck reason about existential quantification required
quite a bit of work to get right. This is also one of the points where our tool
most clearly shows its limitations.

\bibliographystyle{plain}
\bibliography{bibliography}

\end{document}
